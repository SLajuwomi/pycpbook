% =================================================================
% Python Competitive Programming Notebook (PyCPBook)
% Main LaTeX Document
%
% This file is the root of the LaTeX document structure.
% It sets up the document class, loads the preamble and style package,
% and then includes the content from the various chapter files.
%
% NOTE: The chapter files are automatically populated by the build script.
% This file defines the order in which chapters appear in the final document.
% =================================================================

\documentclass[10pt,twocolumn]{report}

% --- Load Preamble and Custom Style ---
% The preamble contains all \usepackage commands and global settings.
% Preamble for PyCPBook
% This file includes all necessary LaTeX packages and global settings.

\usepackage[utf8]{inputenc}
\usepackage[T1]{fontenc}

% Core packages for document structure and appearance
\usepackage{geometry}      % For page layout customization
\usepackage{graphicx}      % For including images
\usepackage{amsmath}       % For advanced math environments
\usepackage{amssymb}       % For math symbols
\usepackage{multicol}      % For multi-column layout
\usepackage{fancyhdr}      % For custom headers and footers
\usepackage{xcolor}        % For color definitions and usage
\usepackage{hyperref}      % For PDF hyperlinks and metadata
\usepackage{titlesec}      % For customizing section titles
\usepackage{titling}       % For customizing the main title
\usepackage{lmodern}       % For a modern-looking font
\usepackage[framemethod=default]{mdframed} % For framed boxes around docstrings

% Package for code listings. Minted provides superior highlighting using Pygments.
% This requires the --shell-escape flag for pdflatex.
\usepackage{minted}

% Map problematic Unicode for pdflatex so minted code blocks compile
% U+216B ROMAN NUMERAL TWELVE → "XII"
% U+0662 ARABIC-INDIC DIGIT TWO → "2"
% U+03C0 GREEK SMALL LETTER PI → math \pi
\DeclareUnicodeCharacter{216B}{XII}
\DeclareUnicodeCharacter{0662}{2}
\DeclareUnicodeCharacter{03C0}{$\pi$}

% --- Hyperref Configuration ---
\hypersetup{
    colorlinks=true,
    linkcolor=blue,
    filecolor=magenta,
    urlcolor=cyan,
    pdftitle={Python Competitive Programming Notebook},
    pdfauthor={PyCPBook Community},
    pdfsubject={Competitive Programming},
    pdfkeywords={python, algorithms, data structures},
    bookmarks=true,
    bookmarksopen=true
}
% The style package defines the custom look and feel of the document.
\usepackage{../doc/tex/pycpbook}

% --- Document Metadata ---
\title{Python Competitive Programming Notebook}
\author{PyCPBook Community}
\date{\today}

% --- Document Body ---
\begin{document}

% The \maketitle command generates the title block based on the metadata above.
% We render it in a single column for better appearance.
\twocolumn[
  \begin{@twocolumnfalse}
    \maketitle
    \begin{abstract}
    This document is a reference notebook for competitive programming in Python. It contains a collection of curated algorithms and data structures, complete with explanations and optimized, copy-pasteable code.
    \end{abstract}
    \vspace{2em}
  \end{@twocolumnfalse}
]

% The \tableofcontents command generates a ToC.
\tableofcontents
\newpage

% --- Chapter Inclusion ---
% Each chapter's content is managed in its own directory and included here.
% The content within each chapter's 'chapter.tex' file will be dynamically
% generated by the build script.

\chapter{Fundamentals}
% This file will include the content for the "Dynamic Programming" chapter.
% It is automatically managed by the build script.

\chapter{Standard Library}
% This file will include the content for the "Dynamic Programming" chapter.
% It is automatically managed by the build script.

\chapter{Contest \& Setup}
% This file will include the content for the "Dynamic Programming" chapter.
% It is automatically managed by the build script.

\chapter{Data Structures}
% This file will include the content for the "Dynamic Programming" chapter.
% It is automatically managed by the build script.

\chapter{Graph Algorithms}
% This file will include the content for the "Dynamic Programming" chapter.
% It is automatically managed by the build script.

\chapter{String Algorithms}
% This file will include the content for the "Dynamic Programming" chapter.
% It is automatically managed by the build script.

\chapter{Mathematics \& Number Theory}
% This file will include the content for the "Dynamic Programming" chapter.
% It is automatically managed by the build script.

\chapter{Geometry}
% This file will include the content for the "Dynamic Programming" chapter.
% It is automatically managed by the build script.

\chapter{Dynamic Programming}
% This file will include the content for the "Dynamic Programming" chapter.
% It is automatically managed by the build script.

\end{document}